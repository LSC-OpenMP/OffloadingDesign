\section{OpenMP 4.0 offloading overview}\label{sc:OpenMP offloading overview}

The OpenMP 4.0 specification introduces offloading directives that can be used to take advantage of accelerators or devices, in OpenMP terminology (see the OpenMP 4.0 specification at www.openmp.org).
\begin{itemize}
  \item \textbf{Device}: an implementation-defined logical execution unit. The execution model is host-centric such that a host device offloads code and data to target devices.

  \item \textbf{Target} regions are structured code blocks that execute on a target device. This is conditional on the run-time availability of a device, the ability of the compiler to generate device code, and other factors. 

  \item \textbf{Target declarations} are used to specify mapping of global variables to a device and create device specific versions of functions that can be called from target regions.

  \item \textbf{Device data environments} contain the variables that are currently present on the target device.

  \item \textbf{A mapped variable} is a variable in a (host) data environment with a corresponding variable in a device data environment.

\end{itemize}

The \dtarget{} directive creates both a device data environment and a target region. It may have associated clauses to specify further details, like the exact device to use if more than one is present in the system (\cdevice{} clause) or whether the data should be moved to/from the device or only allocated in the device memory (\cmap{} clause). The \ddeclaretarget{} directive declares that enclosed global variables and functions should have corresponding device versions.

Example~\ref{ex:OffloadingExample} illustrates how offloading can be expressed using the available set of directives.


\begin{example}
\lstset{basicstyle=\scriptsize,frame=single}
\begin{lstlisting}
// foo() will be implemented for the host and device
#pragma omp declare target
int foo(int[1000]);
#pragma omp end declare target

// All declarations outside a declare target region will NOT be implemented
// for the device 
...
int device_count = omp_get_num_devices();
int device_no;
int *red = malloc(device_count * sizeof(int));
int c[1000];

// Several host threads are going to be spawned to execute the structured 
// block associated with the parallel region (this is unrelated with the
// offloading support)

#pragma omp parallel for
for (i = 0; i < 1000; i++) {

  device_no = i % device_count;

  // The target directive specifies that the execution of associated 
  // structured block (target region) should be transferred to the device. 
  //
  // The device clause states that device whose ID is device_no should be 
  // used.
  // 
  // The first map clause specifies that an instance of c has to be allocated 
  // in the device and updated with the host content of c prior the execution 
  // of the target region (to:).
  //
  // The second map clause specifies that an instance of red[i] has to be 
  // allocated in the device and updated with the host content prior the 
  // execution of the target region and that the host instance should be 
  // updated with the content of the device instance after the target region 
  // execution is complete.
  //
  // If for some reason the device is not available, a host version of the 
  // target region is executed instead.
  #pragma omp target device(device_no) map(to:c) map(red[i])
  { 
    // This code is going to be executed on the device(s)
    red[i] += foo(c); 
  }

  // The execution of the target region is blocking, therefore the dedicated 
  // host thread will wait for the device to complete execution.
}

for (i = 0; i< device_count; i++)
 total_red += red[i];
\end{lstlisting}
\lstset{basicstyle=\small\bfseries,frame=none}
\caption{Offloading expressed with OpenMP directives.}
\label{ex:OffloadingExample}
\end{example}